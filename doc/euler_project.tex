\documentclass[twoside]{article}
\usepackage[utf8]{inputenc}
\usepackage[]{fontenc}
\usepackage{graphicx}

\usepackage{anysize}
\marginsize{3cm}{3cm}{1.5cm}{1.5cm}

\title{Euler Integrál számolása}
\date{\today}
\author{Balázs Tamás}

\begin{document}
	
	\maketitle
	\section*{A program}
		A program explicit Euler-módszer segítségével megoldja a Föld-Hold-rendszer mozgásegyenletét.
		\subsection*{Szimuláció}
		A feladatban síkban dolgozunk úgy, hogy geocentrikus koordinátákat használunk, ezáltal a Föld a 0,0 koordinátapontban van rögzítve.
		
	\section*{Eremények}
		\par Az ábra jól mutatja a Hold Földhöz való mozgását.
		\subsection*{Kapott adatok}	
		\par Az alábbi diagram egy periódust mutat:
		\begin{center}
			\includegraphics[width = \textwidth]{../build/eredmenyek.png}
		\end{center}
\end{document}